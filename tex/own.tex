\documentclass[11pt]{article}
\usepackage[utf8]{inputenc}
\usepackage[german]{isodate}
\usepackage{fullpage}
\usepackage{tabularx}
\usepackage{capt-of}
\usepackage{amsmath}
\title{CS102 \LaTeX\spaceÜbung}
\author{Sein Coray}
\begin{document}
\maketitle
\section{Erster Abschnitt}
Hier könnte ganz viel Text stehen, aber niemand hat Lust so viel Unnötiges zu lesen.
\section{Tabelle}
In der Tabelle finden Sie ganz viel.
\begin{center}
		\begin{tabularx}{0.8\linewidth}{X|X|X|X}
			& Punkte erhalten & Punkte möglich & \% \\
			\hline
			Aufgabe 1 & 2 & 4 & 50 \\
			Aufgabe 2 & 1.5 & 3 & 50 \\
			Aufgabe 3 & 3 & 3 & 100
		\end{tabularx}
		\captionof{table}{Der Student war nicht so fleissig.}
	\end{center}
\section{Formeln}
\subsection{Pythagoras}
	Der Satz des Pythagoras errechnet sich wie folgt: $a^{2} + b^{2} = c^{2}$. Daraus können
	wir die Länge der Hypothenuse wie folgt berechnen: $c = \sqrt{a^{2} + b^{2}}$.

\subsection{Summen}
	Wir können auch die Formel für eine Summe angeben:
	\begin{align}
		s &= \sum_{i=1}^{n}i = \frac{n \cdot (n + 1)}{2}
	\end{align}
\end{document}
